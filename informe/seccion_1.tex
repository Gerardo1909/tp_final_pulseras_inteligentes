\section{Presentación del escenario}
\subsection{Empresa}
Una compañía tecnológica está desarrollando una plataforma integral basada en dispositivos portátiles inteligentes. Estas pulseras monitorean continuamente diversos parámetros biométricos relacionados con la salud y la actividad física de los usuarios. 
Los dispositivos capturan datos como: frecuencia cardíaca, calidad del sueño, niveles de actividad, entre otros. Además, se integran con una aplicación móvil que permite a los usuarios interactuar, visualizar información, recibir recomendaciones personalizadas y gestionar sus suscripciones.
Nosotros como equipo, estamos encargados del área de datos del proyecto. Nuestra metodología de trabajo está basada en Scrum, un tipo de metodología ágil que nos permite iterar sobre la implementación en ciclos de tiempo determinados. En una primera instancia, planeamos llegar a una versión MVP (minimum viable product: mínimo producto viable) para que la empresa analice resultados y pueda tomar decisiones informadas sobre el futuro del producto, y para poder cambiar aspectos clave del mismo y de los datos a consumir por las diferentes áreas.
\subsection{Situación a implementar}
El principal desafío es el diseño y desarrollo de una infraestructura de datos robusta y escalable, que pueda integrar y soportar múltiples fuentes de información. Por un lado, el sistema transaccional que gestiona las suscripciones, pagos y perfiles de usuarios. Por el otro, el flujo constante y masivo (en streaming) de los datos biométricos de los usuarios que se generan por el uso de las pulseras. También resulta fundamental incorporar la información de la interacción de los usuarios con la aplicación móvil: eventos de navegación, uso de funcionalidades, y comportamiento dentro de la plataforma.
\subsection{Justificación}
La solución que requiere la empresa implica unificar y consolidar toda la información detallada anteriormente en un Data Warehouse que funcione como el núcleo central de análisis de datos. Esto permite a la empresa obtener métricas clave y parámetros de desempeño del negocio, tales como tendencias de suscripción. Y tanto a la empresa como a los usuarios, les permite acceder a patrones de uso de las pulseras y la aplicación, así como indicadores de salud agregados.
Con esta implementación contamos con un modelo de datos integrado, que facilitará la generación de predicciones y modelos analíticos avanzados. Estos están orientados a optimizar la experiencia del usuario, mejorar la retención, detectar patrones de salud relevantes, y respaldar la toma de decisiones estratégicas a nivel comercial y operativo. De esta forma, establecemos una infraestructura capaz de soportar tanto análisis históricos como actuales, y la construcción de reportes y dashboards que reflejen el estado y evolución del negocio y la salud de sus usuarios.
