
\section{Presentación de los motores SQL y NoSQL}
\subsection{Proveedores}
Para el desarrollo del proyecto utilizamos los sistemas gestores de bases de datos Supabase (basado en PostgreSQL) y MongoDB Atlas (base de datos NoSQL), ambos ofrecidos como servicios en la nube.
\subsection{Justificar de las elecciones}
Optamos por estos motores debido a su facilidad de acceso y administración, al estar alojados en plataformas cloud que permiten la gestión remota de las bases de datos desde cualquier equipo con conexión a internet, eliminando la necesidad de contar con infraestructura física local (on-premise). Además, estas soluciones brindan escalabilidad y flexibilidad para manejar tanto datos estructurados (relacionales) como datos semi-estructurados o no estructurados, adaptándose así a los diferentes tipos de información que el proyecto requiere.
\subsection{Tipo de licencia}
Ambas plataformas ofrecen planes gratuitos que permiten comenzar el desarrollo sin costo inicial, ideal para el MVP. Tanto Supabase como MongoDB Atlas cuentan con opciones de suscripción pagas que proporcionan mayores recursos, capacidades y funcionalidades avanzadas, adecuándose a las necesidades de crecimiento del proyecto.
\subsection{Cómo conseguirlo, dónde descargarlo si se quiere hacer local}
Los servicios se pueden contratar y configurar directamente en sus portales oficiales:
\begin{itemize}
    \item \href{https://supabase.com/pricing}{Supabase - Planes y precios}
    \item \href{https://www.mongodb.com/pricing}{MongoDB Atlas - Planes y precios}
\end{itemize}
Para quienes deseen un entorno local, PostgreSQL puede descargarse e instalarse de forma gratuita desde su sitio oficial, mientras que MongoDB ofrece versiones Community para uso local disponibles en su portal.
